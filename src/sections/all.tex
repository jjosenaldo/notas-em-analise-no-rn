\section{Análise no $\reals^n$}

\begin{definition}[Espaço Vetorial]
	Um \emph{espaço vetorial} $(E,+,\cdot)$ é tal que as operações
	$+:E\times E \to E$ e $\cdot: \reals \times E \to E$ satisfazem 
	as seguintes propriedades:
	\begin{itemize}
		\item $(E,+)$ é um grupo abeliano;
		\item $1\cdot \vec x = \vec x$;
		\item $a\cdot (b\cdot \vec x) = (a\cdot b)\cdot \vec x$;
		\item $(a+b)\cdot \vec x = a\cdot \vec x + b\cdot \vec x$;
		\item $a\cdot (\vec x + \vec y) = a \cdot \vec x + a \cdot \vec y$.
	\end{itemize}
\end{definition}

\begin{definition}[Produto Interno]
	Seja $E$ um espaço vetorial. Um \emph{produto interno} em $E$ é uma função $\prodi{}{}:E\times E \to \reals$
	que verifica:
	\begin{enumerate}
		\item $\prodiv xy = \prodiv yx$;
		\item $\prodiv{x+y}z = \prodiv xz + \prodiv yz$;
		\item $\prodi{a \vec x}{\vec y} = a\prodiv xy$;
		\item $\prodiv xx \ge 0$, e $\prodiv xx = 0 \iff \vec x = \vec 0$.
	\end{enumerate}
	Um espaço vetorial com produto interno é chamado de \emph{espaço com produto interno}.
\end{definition}

\begin{definition}[Matriz Definida Positiva]
	Uma matriz quadrada $A = (a_{ij})_n$ é \emph{definida positiva} quando, para todo $\vec x \in \reals^n$ não nulo,
	\[\sum\limits_{i=1}^n \sum\limits_{j=1}^n a_{ij}x_ix_j > 0\]
\end{definition}

\begin{fact}
	Dada uma matriz definida positiva $A = (a_{ij})_n$, a função $\prodi : \reals^n \times \reals^n \to \reals$
	dada por
	\[\prodiv xy = \sum\limits_{i=1}^n \sum\limits_{j=1}^n a_{ij}x_iy_j \]
	é um produto interno. Podem-se, então, definir uma quantidade não enumerável de produtos internos no espaço 
	$\reals^n$.
\end{fact}

\begin{definition}[Norma Associada ao Produto Interno]
	A \emph{norma $\normprodi{} : E \to \reals$ associada a um produto interno (NAPI)}
	$\prodi{}{}$
	 de um espaço vetorial $E$ é definida da seguinte forma:
	\[\normprodiv{x} = \sqrt{\prodiv xx}\]
\end{definition}

\begin{theorem}[Desigualdade de Cauchy--Schwarz]
	\[\modu{\prodiv xy} \le \normprodiv x \cdot \normprodiv y \]
	A igualdade é válida sse $\vec x$ e $\vec y$ são LD.
\end{theorem}

\begin{fact}
	\[\normprodi{\vec x + \vec y} \le \normprodiv x + \normprodiv y \]
\end{fact}

\begin{definition}[Norma em um Espaço Vetorial]
	Uma \emph{norma} num espaço $E$ é uma aplicação $\norm\  : E \to \reals$
	satisfazendo o seguinte:
	\begin{enumerate}
		\item $\normv x \ge 0$, e $\normv x = 0 \iff x = 0$;
		\item $\norm{a\vec x} = \modu a \normv x$ (homogeneidade);
		\item $\norm{\vec x + \vec y} \le \normv x + \normv y$ (desigualdade triangular).
	\end{enumerate}
	Um espaço vetorial é \emph{normado} quando ele tem uma norma.
\end{definition}

\begin{fact}
	A NAPI é uma norma.
\end{fact}

\begin{fact}
	Seja $(E,\norm\ )$ um espaço normado. Então 
	\[\modu{\normv x - \normv y} \le \norm{\vec x - \vec y}\]
\end{fact}

\begin{definition}[Norma do Máximo]
	\[\normmaxv x = \max\limits_{1\le i \le n}\modu{x_i}\]
\end{definition}

\begin{fact}
	A norma do máximo é uma norma em $\rn$.
\end{fact}

\begin{definition}[Norma da Soma]
	\[\normsv x = \sum\limits_{i=1}^n \modu{x_i}\]
\end{definition}

\begin{fact}
	A norma da soma é uma norma em $\rn$.
\end{fact}

\begin{definition}[Norma Euclidiana em $\rn$]
	\[\normeucv x = \sqrt{\sum\limits_{i=1}^n x_i^2}\]
	A norma euclidiana também é chamada de \emph{norma usual} em $\rn$.
\end{definition}

\begin{fact}
	A norma euclidiana é uma norma em $\rn$.
\end{fact}

\begin{definition}[Identidade do Paralelogramo]
	Sejam $E$ um espaço normado. A \emph{identidade do paralelogramo} é dada
	por 
	\[\norm{\vec x+\vec y}^2 +\norm{\vec x-\vec y}^2 = 2\prn{\normv x^2+\normv y^2} \]
\end{definition}

\begin{theorem}
	Uma norma é proveniente de um produto interno se, e somente se, ela verifica a 
	identidade do paralelogramo.
\end{theorem}

\begin{fact}
	As normas do máximo e da soma em $\rn$ não satisfazem a identidade do paralelogramo
	e, portanto, não são provenientes de produto interno.
\end{fact}

\begin{definition}[Normas Equivalentes]
	As normas $\norm\ {}_1$ e $\norm\ {}_2$ são \emph{equivalentes} quando existem $c_1,c_2> 0$
	tais que \[c_1 \normv x_1 \le \normv x_2 \le c_2 \normv x_1\] para todos os vetores do espaço.
\end{definition}

\begin{fact}
	Num espaço vetorial de dimensão finita, todas as normas são equivalentes.
\end{fact}

\begin{definition}[Bola Aberta]
	Seja $(E, \norm\ )$ um espaço normado. Uma \emph{bola aberta}  de centro $a$ e raio $ r>0$,
	denotada por $\oball ar$ (ou $\mathrm B_r(a)$), é definida como sendo o seguinte conjunto:
	\begin{displaymath}
		\ball(a,r) = \set{\vec x \in E \st \norm{\vec x - \vec a} < r}
	\end{displaymath}
\end{definition}

\begin{definition}[Bola Fechada]
	Uma \emph{bola fechada}  de centro $a$ e raio $ r>0$,
	denotada por $\cball ar$ (ou $\mathrm B_r[a]$), é definida como sendo o seguinte conjunto:
	\begin{displaymath}
		\cball ar = \set{\vec x \in E \st \norm{\vec x - \vec a} \le r}
	\end{displaymath}
\end{definition}

\begin{definition}[Esfera]
	Uma \emph{esfera}  de centro $a$ e raio $ r>0$,
	denotada por $\sphere[a,r]$ (ou $\sphere_r[a]$), é definida como sendo o seguinte conjunto:
	\begin{displaymath}
		\sphere[a,r] = \set{\vec x \in E \st \norm{\vec x - \vec a} = r}
	\end{displaymath}
\end{definition}

\begin{definition}[Conjunto Limitado]
	Seja $(E,\norm\ )$ um espaço normado e $X \subset E$. Dizemos que $X$ é \emph{limitado}
	quando existir $r>0$ tal que \[X \subset \cball {\vec 0}r\] Do contrário, o conjunto é 
	\emph{ilimitado} (não é limitado).
\end{definition}

\begin{remark}
	Em $\rn$, como todas as normas são equivalentes, para provar que um conjunto é limitado,
	basta limitá-lo em qualquer norma.
\end{remark}

\begin{definition}[Função Limitada]
	Dados $(E, \norm\ _E)$ e $(F, \norm\ _F)$ espaços normados e $f : E \to F$, dizemos que 
	$f$ é $\emph{limitada}$ quando existe $r>0$ tal que, para todo $\vec x \in E$, \[\norm{f(\vec x)}_F < r\]
\end{definition}

\begin{definition}[Segmento]
	Seja $(E, \norm\ )$ espaço normado. Dados $\vec a,\vec b\in E$, definimos o \emph{segmento} 
	(que liga $\vec a$ a $\vec b$) como sendo o conjunto:
	\[[\vec a, \vec b] = \set{\vec z \in E \st \vec z = (1-t)\vec a + t\vec b \text{ para algum }t\in[0,1]}\]
\end{definition}

\begin{definition}[Conjunto Convexo]
	Dizemos que $X \subset E$ é \emph{convexo} quando $[\vec x, \vec y] \subset X$ para todos
	$\vec x,\vec y\in X$.
\end{definition}

\begin{definition}[Ponto Interior]
	Seja $(E,\norm\ )$ espaço normado, e $X \subset E$. Dizemos que $a \in X$ é \emph{ponto interior}
	a $X$ quando existe $r_a > 0$ tal que $\oball a{r_a} \subset X$. 
\end{definition}

\begin{definition}[Interior de um Conjunto]
	O conjunto dos pontos interiores de $X \subset E$ é chamado de \emph{interior} de $X$
	é denotado por $\interior X$.
\end{definition}

\begin{fact}
	\begin{enumerate}
		\item[]
		\item $\interior X \subset X$
		\item $X \subset Y \implies \interior X \subset \interior Y$
	\end{enumerate}
\end{fact}

\begin{definition}[Conjunto Aberto]
	Dizemos que $A \subset E$ é \emph{aberto} quando todo ponto do conjunto $A$
	é ponto interior a $A$, isto é, $\interior A = A$.
\end{definition}

\begin{fact}
	Bolas abertas são conjuntos abertos.
\end{fact}

\begin{fact}[Propriedades de Abertos]
	Seja $(E, \norm \ )$ um espaço normado. Então 
	\begin{enumerate}
		\item Se $X$ e $Y$ são abertos em $E$, $X\cap Y$ é aberto em $E$.
		\item Se $(X_\lambda)_{\lambda \in L}$ é uma família de conjuntos abertos, então
		$\bigcup\limits_{\lambda \in L}X_\lambda$ é um conjunto aberto.
	\end{enumerate}
\end{fact}

\begin{definition}
	Seja $(E,\norm\ )$ um espaço normado, $X \subset E$ e $a \in E$. Dizemos que $a$
	pertence à \emph{fronteira} de $X$, e denotamos $a \in \partial X$ (ou $a \in \border X$) quando 
	$a \notin \interior X$ e $a \notin \interior(\compl X)$.
\end{definition}

\begin{fact}
	Dado $E$ espaço normado e $X \subset E$, então o conjunto 
	$\set{\interior X, \interior(\compl X), \border X}$ é uma partição de $X$.
\end{fact}

\begin{fact}[Propriedades da Fronteira]
	Sejam $E$ um espaço normado e $X \subset E$.
	\begin{enumerate}
		\item $X \subset \interior X \cup \border X$.
		\item Se $\border X = \emptyset$, então $X$ é aberto.
	\end{enumerate}
\end{fact}

\begin{definition}[Abertos Relativos]
	Sejam $E$ espaço normado e $A \subset X \subset E$. Dizemos que 
	$A$ é aberto em $X$ (aberto \emph{relativo} a $X$) quando dado $a \in A$,
	existe $r > 0$ tal que $\oball ar \cap X \subset A$.
\end{definition}

\begin{theorem}
	Sejam $E$ espaço normado e $A \subset X \subset E$. Então 
	$A$ é aberto em $X$ sse existe $U \subset E$ aberto tal que 
	$A = U \cap X$.
\end{theorem}

\begin{fact}
	Sejam $E$ espaço normado e $A \subset X \subset E$ com $X$ aberto. 
	Então $A$ é aberto em $X$ sse $A$ é aberto.
\end{fact}